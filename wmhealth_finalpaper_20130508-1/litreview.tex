\fontsize{12bp}{14bp}\selectfont
In "Supersizing Supercenters? The Impact of Wal-Mart Supercenters on Body Mass Index", Professor Courtmanche at the University of North Carolina notes that prior research has linked technological progress and the rise of obesity.  As the overall price of food drops, the consumption of food increases.  Because Wal-Mart has invested heavily in logistics and has strong pricing power with its vendors, he reasons that we should find evidence that obesity rates rise in a location after a Wal-Mart store opens.  Indeed, he found that the average BMI rises by 0.25 units and the obesity rate rises by 2.4\% within ten years of the store opening.  Further he finds that approximately 11\% of the rise in obesity since the late 1980s can be attributed to the proliferation of Wal-Mart stores.\footnote{Charles Courtemanche. Supersizing Supercenters? The Impact of Wal-Mart Supercenters on Body Mass Index and Obesity. Journal of Urban Economics, Elsevier, vol. 69(2), pages 165-181, March.}