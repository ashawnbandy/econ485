\fontsize{12bp}{14bp}\selectfont
\subsection{Data Overview}
We used county level data for the United States, excluding counties in Virginia, Alaska, and Hawaii.  All of the variables that were percentages take on values that are between 0 and 100, not between 0 and 1. Our single dependent variable {\textit{pct\_obese}} (Percentage of adult obesity in a county) came from the University of Wisconsin�s 2013 County Health Rankings. Our primary explanatory variable was {\textit{wm_pertenthou}} (Walmart stores per 10,000 residents in a county). The location for each Walmart came from GPS Points-of-Interest data. The variable {\textit{ wm_per_sq }} is the squared term of our primary explanatory variable; it measures if there is an increasing or decreasing marginal effect. Data from the Bureau of Economic Analysis was used to control for employment factors. We used 2011 estimates from the U.S. Census Bureau to control for demographic factors such as education, age, gender, and ethnicity. See tables A.1 and A.2 in appendix for a complete list of all of our explanatory variables and their summary statistics, respectively.