\fontsize{12bp}{14bp}\selectfont
We take our results as a preliminary indicator that there exists a relationship between the presence of Wal-Mart stores and a local rise in obesity.  Our findings are supported by Courtmanche\footnote{Charles Courtemanche. Supersizing Supercenters? The Impact of Wal-Mart Supercenters on Body Mass Index and Obesity.
Journal of Urban Economics, Elsevier, vol. 69(2), pages 165-181, March.}.  We also agree with his conclusion that the main effect in play is the pricing power Wal-Mart wields over its vendors, particularly with industrial, processed foods.  As traditional grocers in a community are replaced, the aggregate diet shifts because these foods - which are linked with obesity - are more readily substituted against fresh produce.  As retail consumers, we are similar to foragers in the sense that we take from our surroundings what is most readily available.  But as Courtmanche writes, Wal-Mart is not evil.  Instead, the firm supplies what it perceives as demand by its customers.  We hope to contribute to the larger dialog a sense that we are not merely passive foragers because we have the ability to recognize hidden externalities.

The scope of our research is limited to the set of tools we acquired in the course of our first semester of econometrics and by the time allotted to explore and document our findings.  We suggest that our next steps would begin by resolving the degree to which the relationship between Wal-Mart and obesity is merely correlation.  We recognize the possibility that these two phenomenon arose independently in the southern United States: for example, they may both be products of the rapid industrialization of the South in the last half the of the 20th century.  We would also like to quantify the direct and indirect costs (i.e. increased medical costs), if any, associated with obesity that arises from the proliferation of Wal-Mart stores.  