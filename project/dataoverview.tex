\fontsize{12bp}{14bp}\selectfont
\subsection{Data Overview}
We used county level data for the United States, excluding counties in Virginia, Alaska, and Hawaii. All of the variables that were percentages take on values that are between 0 and 100, not between 0 and 1. Our single dependent variable {\textbf{pct_obese}} (Percentage of adult obesity in a county) came from the University of Wisconsin's 2013 County Health Rankings. Our primary explanatory variable was {\textbf{wm_pertenthou}} (Walmart stores per 10,000 residents in a county). The location for each Walmart location came from GPS Point-of-Interest data. The variable {\textbf{wm_per_sq}} is the squared term of our primary explanatory variable; it measures if there is an increasing or decreasing marginal effect. Data from the Bureau of Economic Analysis was used to control for employment factors such as per capita income. We used 2011 estimates from the U.S. Census Bureau to control  for demographic factors such as household income, poverty rates, travel time to work, education, age, gender, and ethnicity. Lastly, data for other control factors such as unemployment, fast food restaurants, excessive alcohol consumption, smoking, and being uninsured came from University of Wisconsin's 2013 County Health Rankings.