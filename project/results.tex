\fontsize{12bp}{14bp}\selectfont

The high R-squared and adjusted R-squared (0.64 and 0.63, respectively) indicate that the
model explains a significant portion of the variation in the pct_obese variable. The Breusch-
Pagan test found some heteroskedasticity in the original model, so the regression was
executed with robust standard errors. The VIF for each variable were also analyzed to detect
multicollinearity, finding that four variables had VIF values above ten, but the average VIF was
5.90. Based on these results, we conclude that multicollinearity is not a serious problem in the
model. 
\fontsize{12bp}{14bp}\selectfont

We observe that both Wal-Mart variables are statistically significant, finding that for an
additional Wal-Mart store per 10,000 people, there is a 0.89\% increase of the percentage of
obese in the county, but a 0.45\% decreasing marginal return. We also observe that the other
significant variables in the model match our expectations regarding their relative effects on
obesity (e.g. education is negatively correlated, while poverty and access to fast good are
positively correlated).