\item Use your results from L1 for this problem.
	\begin{enumerate}[a.]
		\item Is there a problem with heteroskedasticity? Explain how you know.
		\item For each of the models below (which correspond to the models from part b of L1, interpret the coefficients identified here.
			\begin{enumerate}[i.]
				\item Dependent variable: wage; explanatory variables are female nonwhite union education exper. Interpret the coefficient on exper
				\item Dependent variable: annual\_wage; explanatory variables are female nonwhite union education exper. Interpret the coefficient on exper 
				\item Dependent variable: lnwage; explanatory variables are female nonwhite union education exper Interpret the coefficient on exper
				\item Dependent variable: lnwage; explanatory variables are female nonwhite union education lnexp. Interpret the coefficient on lnexp
				\item Dependent variable: lnwage; explanatory variables are female nonwhite union education exper exp\_squared - Interpret the effect of experience on wages, using exper and exp\_squared 
				\item Dependent variable: lnwage; explanatory variables are female nonwhite union college some\_college exper - Interpret the effect of having at least 4 years of college on wages
			\end{enumerate}
		\item Why did we leave no\_college out of the final model (estimated in L1, parts b, vi)?
		\item In L1, parts c, d, and e you used STATA to calculate an F-statistic. Write out the formula for that F-statistic here. What is the null hypothesis for this F-test? If the critical F-value with 2, 1282 degrees of freedom at the 5\% level is 3.00, would you reject the null hypothesis? What is your conclusion here?
		\item Is the test in L1 part f the same as the test in part e? What is your conclusion based on the result in part f?
		\item In L1 part g and h, you created a new variable and ran a regression on that new variable. What is the null hypothesis you are testing? Based on that hypothesis, demonstrate (show) how we transform our original equation (L1 part b, vi) to the one in L1 part h.
		\item Based on the results from L1 part h, what is your conclusion about the hypothesis you are testing (in other words, the hypothesis you outlined in Q2, part d)?
		\item What does the test in L1 part i tell us? Why couldn�t we calculate this particular test using scalars?
		\item Is there a problem with multicollinearity in the model in L1 part b iv? How do you know?
	\end{enumerate}